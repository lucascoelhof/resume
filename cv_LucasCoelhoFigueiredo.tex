%------------------------------------\dfrac{num}{den}----------------
%	PACKAGES AND OTHER DOCUMENT CONFIGURATIONS
%----------------------------------------------------

\documentclass[12pt, a4paper]{resume} % Use the custom resume.cls style

\usepackage[left=10mm, textheight = 290mm, top=10mm,  bottom=10mm, right=10mm]{geometry} % Document margins
\usepackage[brazilian]{babel}
\usepackage[utf8]{inputenc}
\usepackage{lmodern,textcomp}
\usepackage{xcolor}% http://ctan.org/pkg/xcolor
\usepackage{hyperref}% http://ctan.org/pkg/hyperref
\hypersetup{
	colorlinks=true,
	linkcolor=blue!50!red,
	urlcolor=blue!70!black
}



\name{Lucas Coelho Figueiredo} % Your name
\address{lucascoelhof@gmail.com \\ https://github.com/lucascoelhof}  % Your phone number and email

\begin{document}

%------------------------------------------------
%	EDUCATION SECTION
%------------------------------------------------

\begin{rSection}{Education}

	{\bf University of São Paulo}  \hfill{05/2022 - 04/2024} \\
	MBA in Project  Management \hfill{Online}

	{\bf Federal University of Minas Gerais}  \hfill{08/2015 - 06/2018} \\
	MSc in Electrical Engineering, emphasis in Control and Robotics \hfill{Belo Horizonte, Brazil} \\
	Thesis: Human-robot swarm interaction on multi-robot coverage control with Virtual Reality

		{\bf Udacity}  \hfill{01/2019 - 06/2019} \\
	Self-driving car Engineer Nanodegree \hfill{Online}

	{\bf Rockstart Accelerator}  \hfill{08/2016 - 08/2016} \\
	Startup Acceleration Program, intensive course on entrepreneurship \hfill{Amsterdam, The Netherlands}

	{\bf Federal University of Minas Gerais}  \hfill{03/2010 - 07/2015} \\
	Bachelor in Control and Automation Engineering \hfill{Belo Horizonte, Brazil}

	{\bf The University of Texas at Austin} \hfill {08/2013 - 05/2014} \\
	Exchange Program in Electrical Engineering and Computer Science \hfill {Austin, USA}

	{\bf Federal Center of Technological Education} \hfill {05/2007 - 12/2009} \\
	High School and Technical Program in Electronics \hfill{Belo Horizonte, Brazil}

\end{rSection}

%-------------------------------------------------
%	WORK EXPERIENCE SECTION
%-------------------------------------------------


\begin{rSection}{Experience}

	\begin{rSubsection}{Realtime Robotics Inc.}{12/2022 - Today}{\normalfont Senior Software Engineer}{ \normalfont Boston, USA (Remote)}
		\item C\texttt{++} engineer on the RapidPlan Create products
		\item Implemented novel motion planning algorithms for robotic arm obstacle avoidance that converge 60\% faster and with 20\% shorter paths than previous implementations
		\item Scrum Master of the team, responsible for the team's ceremonies and processes
	\end{rSubsection}

	\begin{rSubsection}{Hexagon Mining}{11/2017 - 11/2022}{\normalfont Specialist Software Engineer III - Autonomous Vehicles}{ \normalfont Belo Horizonte, Brazil}
		\item Technical lead on the fleet management system for autonomous vehicles, helping the team designing highly reliable machine-to-machine communication protocols for distributed systems in mining environment using DDS in C\texttt{++}
		\item Responsible for the robotics framework in C\texttt{++} for Hexagon Mining autonomous solution
		\item Great experience integrating complex products into Hexagon's autonomous solutions, including a high precision GPS-based parking assist, a fleet management system for mining, and a collision-avoidance system
		\item Implemented highly reliable machine-to-machine communication protocols using DDS in C\texttt{++} (RTI and OpenSplice) and integrated with three different OEMs
		\item Technical mentorship of other seven engineers in the team

	\end{rSubsection}

	\begin{rSubsection}{Freelance Android Developer}{10/2016- 05/2019}{\normalfont}{ \normalfont Belo Horizonte, Brazil}
		\item Designed portable energy meter circuit boards, choosing microcontrollers, sensors and communication protocols, while also developing the Android app to control it
		\item Developed Android apps for CEMIG and Minipa, raising requirements with clients, and developed the backend in Java
	\end{rSubsection}

	%\begin{rSubsection}{PPGEE - UFMG}{09/2015 - 11/2015}{\normalfont Teaching Assistant}{ \normalfont Belo Horizonte, Brazil}
	%	\item Working as a Teaching Assistant on the graduate course "Fundamentals and Applications of Fuzzy Systems"
	%	\item Coordinated two groups of students, helping them to implement fuzzy controllers in two non-linear systems: a twin rotor, and a magnetic levitator
	%\end{rSubsection}

	\begin{rSubsection}{Newatt Energy Systems}{09/2015 - 08/2017}{\normalfont Co-Founder and CTO}{ \normalfont Belo Horizonte, Brazil, and Amsterdam, The Netherlands}
		\item Designed embedded hardware and firmware for STM32 microcontrollers in a wireless energy sensor, using C\texttt{++} and Python
		\item Responsible for the technical decisions and relationship with partners and investors
		\item Managed the team for the implementation of the cloud architecture for energy data processing in Node.js and Python on AWS. Implemented data collection using MQTT
	\end{rSubsection}

	%\begin{rSubsection}{Equalizar}{08/2015 - 02/2016}{\normalfont Teacher}{ \normalfont Belo Horizonte, Brazil}
	%	\item Equalizar is a non-profit organization run by volunteers that helps low-income students from public schools to get accepted on the best universities
	%	\item Worked as a volunteer teacher in Math and teaching assistant in Physics
	%\end{rSubsection}

	\begin{rSubsection}{DTI Sistemas}{08/2014 - 08/2015}{\normalfont Software Development Intern}{\normalfont Belo Horizonte, Brazil}
		\item Maintenance and development for applications in Java and C\texttt{\#} in various industries such as car rental, dam management and healthcare
	\end{rSubsection}

	\begin{rSubsection}{Multi-robot Systems Laboratory - Boston University}{05/2014 - 08/2014}{\normalfont Summer Research Intern: 400 hours}{\normalfont Boston,  USA}
		\item Created algorithms for multi-robot systems applied in area coverage and autonomous exploration
		\item Developed embedded software for \textit{m3pi} robots in C\texttt{++}
		\item Published academic results at ICRA 2015 and IJRR 2017
	\end{rSubsection}

	\begin{rSubsection}{Computation and Robotics Lab - UFMG}{10/2010 - 08/2013}{\normalfont Undergraduate Research Assistant: 2880 hours}{ \normalfont Belo Horizonte, Brazil}
		\item Developed algorithms for multi-robot coverage and UAV trajectory tracking using C\texttt{++}, ROS, and MATLAB in a Linux environment
		\item Large experience in ROS, working on it since Diamondback version (2011) and becoming a reference on ROS for labmates and professors
	\end{rSubsection}
\end{rSection}




\begin{rSection}{Awards}

	\begin{rSubsection}{Hexagon Mining Hackathon}{2022}{Winner}{ \normalfont Belo Horizonte, Brazil}
		\item Integrated 3 different Hexagon Mining products into a simulator with virtual reality. Got a Xbox Series X as a prize.
	\end{rSubsection}

	\begin{rSubsection}{Sabiá Laranjeira Award}{2017}{Winner}{ \normalfont São Paulo, Brazil}
		\item Awarded by "Plataforma Liderança com Valores" institute for the work on Newatt Energy Systems.
	\end{rSubsection}

	\begin{rSubsection}{Rockstart Smart Energy Program}{2016}{Finalist}{ \normalfont Amsterdam, The Netherlands}
		\item Awarded with a 6-month acceleration program in Amsterdam and EUR 75k in investment.
	\end{rSubsection}

	\begin{rSubsection}{Brazil Scientific Mobility Program}{08/2013 - 08/2014}{Awardee}{ \normalfont Austin, USA}
		\item Selected for a grant that covered a year of studies at The University of Texas at Austin, including tuitions, housing, meals and travel expenses.
	\end{rSubsection}

\end{rSection}

\begin{rSection}{Publications}

	\begin{rSubsection}{Brazilian Automatics Congress}{08/2018}{}{}

		\item L. C. Figueiredo, I. L. Carvalho, L. C. A. Pimenta. ``Voronoi Multi-Robot Coverage Control in Non-Convex Environments with Human Interaction in Virtual Reality''
		\item Experimental results at https://www.youtube.com/watch?v=cpniwb6UrF8

	\end{rSubsection}

	\begin{rSubsection}{International Journal of Robotics Research (IJRR)}{02/2017}{}{}

		\item A. Pierson, L. C. Figueiredo, L. C. A. Pimenta, and M. Schwager. ``Adapting to Sensing and Actuation Variations in Multi-Robot Coverage''

	\end{rSubsection}

	\begin{rSubsection}{IEEE International Conference on Robotics and Automation (ICRA)}{05/2015}{}{ \normalfont Seattle, USA}

		\item A. Pierson, L. C. Figueiredo, L. C. A. Pimenta, and M. Schwager. ``Adapting to Performance Variations in Multi-Robot Coverage''
		\item Experimental results at https://www.youtube.com/watch?v=qyYt3frZ7aw


	\end{rSubsection}


\end{rSection}



%----------------------------------------------------
%	TECHNICAL STRENGTHS SECTION
%----------------------------------------------------

\begin{rSection}{Skills}

	\begin{tabular}{ @{} >{\bfseries}l @{\hspace{6ex}} l }
		Programming & Proficient: C\texttt{++}, GTest, Python, ROS, RTI DDS, Git, Ubuntu          \\
		            & Average: Android, Java, OpenSplice DDS, C\texttt{\#}, CMake, Docker, OpenCV \\
		            & Beginner: Node.js, MATLAB, SQL, PCL, SQL, Keras                             \\
		Office      & Word, PowerPoint, Excel                                                     \\
		Managerial  & Scrum, Agile, Lean Startup                                                  \\
		Other       & LaTeX, UML                                                                  \\
		English     & Fluent                                                                      \\
		Portuguese  & Native                                                                      \\
		Hobbies     & Photography, scuba diving, cooking, mountain biking, electronics projects
	\end{tabular}

\end{rSection}

\end{document}
